\documentclass{banko}
\usepackage{graphicx} % Required for inserting images
\usepackage{lipsum}
\usepackage{amsmath}
\usepackage{amssymb}
\usepackage{ulem}

\title{晩子 Banko Template with International Support with a to long title}
\BPNauthor*{Duy Nam Schlitz}{Some Developer for Low-Level and High-Level, Games, Firmware, Robotics, Medical, Algorithm and Deep Learning; ISAC and ICULS Founder, Deparment for Developing Cosmolang and Cosmo Lingua / Token Programming Language. Independent Researcher at Institute of ISAC Sciences as Biologist, Chemist and Physics}{duynamschlitz@gmail.com}
\date{January 2025}
\BPNundertitle{Roman, CJK, \textcolor{purple}{Cyrillic}, Greek, \textcolor{purple}{\small Arabic, Devanagari, Hebrew, Tamil, Thai, Bengali, Georgian, Armenian, Ethiopic (Ge'ez), Cherokee, Sinhala, Mongolian, Khmer, Tibetanm}}

\BPNyear{2025}
\BPNlocation{Frankfurt}
\BPNdates{22.01}
\BPNcopyright{2025}{Duy Nam Schlitz}
\BPNarchive{209CMa3n9}
\BPNmode{1}
\BPNRevision{1}
\BPNVersion{2-3.1}

\BPNLogo{
\begin{center}
	\begin{minipage}{0.33\textwidth}
		\centering
		\includesvg[width=\textwidth]{IndeResearch.svg}
	\end{minipage}
	\begin{minipage}{0.33\textwidth}
		\centering
		\includesvg[width=\textwidth]{Insti.svg}
	\end{minipage}
\end{center}
}

\abstract{Hello. \ruby{こんにちは}{kon'nichiwa}. \ruby{안녕하세요}{Annyoung-haseyo}. \ruby{你好}{n\u{i} h\u{a}o}. \ruby{Привет}{Privet}. \ruby{مرحبًا}{Marhaban}. \ruby{שלום}{Shalom}. \ruby{Γειά}{Ya}. \ruby{नमस्ते}{Namaste}. \ruby{สวัสดี}{Sawasdee}. Bonjour. Hallo. \ruby{Xin chào}{Sin chào}.  Hola. Ciao.  \ruby{Olá}{Olah} (Portuguese). Salve. Merhaba. Habari. Kumusta. \ruby{Zdravo}{Zdravo}. \ruby{Bok}{Bok}. Selamat siang. }

\bibliography{.bib}

\begin{document}

\maketitle

\romanPage

\tableofcontents

\lipsum[1-8]\sidenote{Welcome to the internet of things}

\clearpage
\arabicPage

\newpage

\section{Introduction}

The 晩子 is a derivitive from Sazuko サズコ and is built for more compactibility with more language and for a more elegant design. I would preciate if this will be use by many of you. I want further expand this project, so if you are interested please contact me by my email (in the authors section).

This showcase use (almost) everything that is to offer with this template by presenting you different things.

\section{Languages}

We can use different languages to write and also different scripts. Like 日本語 or 한국인.

日本語も利用可能で、\LaTeX で使用できます。ちょっとおしゃべりしましょう。昔々、コミちゃんという男の人がいました。彼女は毎日学校に行って俳句を書いていました。私は彼女を愛していましたが、彼女はもう亡くなり、私のあらゆる努力にもかかわらず、二度と彼女に会うことはありません。彼女は世界へ、そしてその先へと旅立った。

한국어도 가능하며 \LaTeX 와 함께 사용할 수 있습니다. 잠깐 이야기를 나눠보죠. 옛날 옛적에 코미찬이라는 남자가 살았습니다. 그녀는 매일 학교에 가서 하이쿠를 썼습니다. 나는 그녀를 사랑했지만, 이제 그녀는 떠났고 내 모든 노력에도 불구하고 나는 그녀를 다시 볼 수 없을 것이다. 그녀는 세상과 그 너머로 나아갔습니다.

韓語可用並可與 \LaTeX 一起使用。我們聊一會兒吧。從前,有一個名叫小見そそ的男人。她每天去上學並寫俳句。我愛她,但現在她已經走了,儘管我盡了一切努力,但我再也見不到她了。她走向了外面的世界。

For non cloud services, please check that every of the CJKVZable character can be display.

Services like overleaf.com do it for you.

\subsection{Ruby}

Ruby: \ruby{日本語}{にほんご}

\subsection{Europe}

Ένα ελληνικό κείμενο: Τι ήταν κάποτε και τι θα είναι κάποτε; Γιατί ο κόσμος μας αποτελείται από τα μικρότερα συστατικά, τα άτομα;\sidenote{Greek}

\importantsection{An important thing to say}

I thank you that you have read it through.

\cite{Brown2022,Davis2014,Chen2015,Harrison2016}
\cite{Brown2022,Chen2015,Davis2014,Harrison2016,Kim2018,Miller2017,Morris2013,Smith2021,Taylor2019,Williams2020}

\cite{Brown2022}

\togglelayout

\lipsum[5-12]

\part{part}

\tableofcontents

\section{section}
\subsection{subsection}
\subsubsection{subsubsection}
\paragraph{paragraph}
\subparagraph{subparagraph}
Привет

私はこの世界と世界の正しい答えを自分自身で創造します。私はこの世界の神々に属します。私は創造主であり神である。世界は私の足元にある。

\lipsum[12]\sidenote{12 Abs.}

\section{Offline: TeXStudio}

Use Biber and XeLaTeX

\printbibliography{}

\sidenotestable

\end{document}